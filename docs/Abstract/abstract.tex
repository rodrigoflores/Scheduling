\documentclass[a4paper,12pt,notitlepage]{article}
\usepackage[T1]{fontenc}
\usepackage[brazil]{babel}
\usepackage[utf8x]{inputenc}
\usepackage{graphics}
\usepackage{times}
\usepackage{ucs}
\usepackage{url}



\title{Resumo do projeto do Curso de Introdução ao Escalonamento \\ Geração de tabelas de futebol}


\author{Rodrigo L. M. Flores \\ \#USP 5127470}


\date{\today}



\begin{document}


\maketitle

Torneios de futebol hoje são muito mais que apenas competições: geram renda e entretenimento em uma escala mundial. Muito dinheiro
hoje está agregado a eles e patrocinadores investem milhões todo ano em troca de terem suas marcas divulgadas pelo mundo todo. Estes
torneios também criam torcedores fanáticos e exagerados que em muitos lugares acabam travando disputas violentas de torcidas em vários 
lugares do mundo, assim como problemas relacionados a isso como por exemplo o vandalismo.

Uma maneira de tentar evitar esse tipo de confronto é gerar uma tabela de campeonato que evite
dois jogos na mesma cidade (ou em cidades bem próximas) no mesmo dia. Outro requisito interessante que uma tabela pode ter é de evitar
ter jogos seguidos dentro ou fora de casa. Um artigo interessante que fala sobre este último assunto é o %Colocar o artigo do futebol holandês
que propõe um algoritmo de geração de tabelas ``sem-time'' (i.e. só são gerados os padrões, os times para cada padrão 
são atribuídos depois). Este algoritmo gera padrões sem quebra\footnote{Chamamos de padrão com quebra quando acontecem dois jogos seguidos
dentro ou fora de casa} para um número ímpar de times e $2n - 2$ quebras para um número par de times, sendo $n$ o número de times.

Com este algoritmo implementado, o objetivo deste trabalho será otimizar a atribuição de times para os padrões segundo alguns critérios. 
Algumas idéias de critério que poderão por ventura ser combinadas, são:

\begin{itemize}
	\item Como rodadas são feitas normalmente em 2 dias, queremos evitar que mais que 2 jogos na mesma rodada sejam na mesma cidade. 
	\item Equipes não podem enfrentar seguidamente equipes de níveis parecidos. 
	\item Minimizar a distância percorrida pelos times no campeonato.
\end{itemize}

Todo código fonte desenvolvido, assim como os testes, monografias e resumos 
pode ser encontrado no repositório do projeto através do endereço \url{http://github.com/rodrigoflores/Scheduling}.

\end{document}
