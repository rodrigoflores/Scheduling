\documentclass[a4paper,12pt,titlepage]{article}
\usepackage[T1]{fontenc}
\usepackage[brazil]{babel}
\usepackage[utf8x]{inputenc}
\usepackage{graphics}
\usepackage{times}
\usepackage{ucs}
\usepackage{url}
\usepackage{algorithmic}
\usepackage{algorithm}


\title{Monografia do projeto do Curso de Introdução ao Escalonamento \\ Geração de tabelas de futebol}


\author{Rodrigo L. M. Flores \\ \#USP 5127470}


\date{\today}



\begin{document}

\maketitle

\section{Introdução}

Torneios de futebol há muito tempo deixaram de ser apenas mero entretenimento para se tornar espetáculos televisivos,
com audiências em níveis bastante elevados e torcidas ocupando cada cadeira do estádio. Estes espetáculos movimentam
alguns bilhões de reais no mundo, seja em patrocínios, transferências de jogadores, arrecadação da venda de ingressos,
venda de produtos licenciados por clubes, entre outras coisas. 

Torcedores se tornam apaixonados pelos seus clubes e a rivalidade com torcedores de outro clube é tão grande a 
ponto de os confrontos com torcedores de uma torcida 
adversária formarem um ambiente semelhante ao de uma guerra, vandalizando o ambiente em sua volta e deixando  
torcedores mortos e feridos.

Embora este problema seja bem comum no Brasil, em outros países como Argentina, Inglaterra, Escócia, Turquia e Itália também costumam
ter problemas parecidos podendo (ou não) ter um plano de fundo político, étnico ou religioso. Por exemplo, os times \textit{Rangers} 
e \textit{Celtic} da cidade de \textit{Glasgow}, Escócia acabam por incluir nesta briga o ódio já existente 
entre protestantes e católicos. 

A questão televisiva também é importante: normalmente não é permitido que a televisão aberta transmita um jogo para a cidade 
na qual acontecerá o jogo. Isso incentiva torcedores a irem ao estádio e acompanharem ao vivo as partidas do seu time. Exceções
costumam acontecer em jogos finais nos quais os estádios normalmente estão lotados e o jogo tem muita audiência. 

Sabendo disso, é necessário gerar tabelas de torneio que levem em conta esse aspecto de não permitir que dois times da mesma cidade joguem 
no mesmo dia, na mesma cidade. Como algumas cidades, como São Paulo e Rio de Janeiro, há mais de dois times, 
é necessário que as rodadas sejam divididas em pelo menos dois dias.  Se chamarmos o número máximo de equipes de uma mesma cidade
de $M$, o número de dias que uma rodada deve durar deve ser $\lceil\frac{M}{2}\rceil$. Felizmente, o número de times de uma
mesma cidade dificilmente passa de $4$, o que permite que rodadas durem apenas $2$ dias (um fim de semana ou uma quarta e quinta).

O objetivo deste trabalho é: dado times e suas cidades, gerar uma tabela para um campeonato de pontos corridos (isto é, todos os times
se enfrentando em jogos de ida e volta) sem que dois times joguem na mesma cidade em uma mesma rodada em um mesmo dia, 
ou, especificamente, sem que mais de $2$ times joguem como mandante na mesma cidade na 
mesma rodada (considerando uma rodada durando $2$ dias). 


\section{Possíveis abordagens}

Uma das possíveis abordagens para a solução deste problema é gerar todas as tabelas possíveis e buscar as que atendem ao nosso critério. 
Como o número de combinações para $20$ equipes (da ordem de $10^{130}$) é gigantesco, podemos rapidamente descartar esta solução. Uma outra abordagem
seria gerar um padrão de jogos e fazer uma busca nestas soluções: gerar todas as combinações é da ordem de $n!$, que para $n$ pequeno (menor que $25$)
é bem possível fazer este cálculo. 

Como muitos times não tem esse problema por eles não terem outros times da mesma cidade disputando o mesmo campeonato, podemos pensar em outra abordagem:
resolver primeiro os times que têm esse problema, e as vagas que sobrarem serão sorteadas para os outros times. Assim garantimos uma solução mais rápida.

\section{Algoritmo de geração de padrões}

O algoritmo, retirado do artigo \cite{90Schreuder}, gera os padrões de jogos (chamados no artigo de \textit{HAP}, 
\textit{Home Away Pattern}) da seguinte maneira: dado um jogo gerador 
(que é o jogo do time $i$ com o $i+1$ na rodada $i$) ele gera os outros jogos da rodada da seguinte maneira:

\begin{algorithm}
\caption{Algoritmo para geração das rodadas}
\label{HAP}
\begin{algorithmic}
\STATE $invert \leftarrow TRUE$
\STATE $round \leftarrow [generatorGame]$
\FOR{$i = 1$ to $\lfloor N/2 \rfloor$}
	\STATE $team1 \leftarrow normalize(generatorGame.home + i)$
	\STATE $team2 \leftarrow normalize(generatorGame.away-i)$
	\IF{$invert$}
		\STATE $round.append((team2,team1))$
		\STATE $invert \leftarrow FALSE$
	\ELSE
		\STATE $round.append((team1,team2))$
		\STATE $invert \leftarrow TRUE$
	\ENDIF
\ENDFOR
\end{algorithmic}
\end{algorithm}

A rotina $normalize$ apenas tira o módulo $N$ e adiciona $1$ de modo a sempre termos um número no intervalo $1$ a $N$. 
O \ref{HAP} resolve o problema para um número ímpar de times, já para um número par precisamos fazer uma artimanha:
resolvemos para um número $N-1$ de times e o time que sobrou joga com o time $N$, que alterna entre jogos em casa e 
jogos fora de casa. 

Uma quebra de padrão \textit{Home-Away} é quando um time joga duas partidas seguidas fora ou dentro de casa. Para
um número ímpar de times, o algoritmo tem $0$ quebras de padrão \textit{Home-Away}, já para um número par de times
acontecem $2N-2$ quebras, o que pode não ser bom mas ainda assim é aceitável. A prova desta propriedade
pode ser vista no artigo \cite{90Schreuder}.

\subsection{Propriedade especial do algoritmo \textit{HAP}}

\bibliographystyle{amsalpha}
\bibliography{bibliografia}

\end{document}
